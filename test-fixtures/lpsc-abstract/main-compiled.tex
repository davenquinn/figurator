The stratigraphy at northeast Syrtis Major {[}Figure
\ref{fig:overview-map}{]} provides a unique temporally-constrained
record of changing hydrological conditions during the
Noachian--Hesperian transition on early Mars \autocite{Ehlmann2012}. The
stratigraphy, bracketed by the mid-Noachian Isidis Basin-forming impact
and mid-Hesperian Syrtis Major lavas \autocite{Hiesinger2004}, contains
layered sulfates superposed atop Noachian basement units altered to
carbonates and clays. The sulfate unit is recessive and layered at
meter-scale, with some areas containing boxwork polygonal ridges. It is
capped by lava flows contiguous with the Syrtis Major volcanic province.

The upward progression from alkaline to acidic aqueous environments (and
ultimately to capping lavas with no evidence of pervasive alteration) is
characteristic of the planet's first billion years
\autocites[e.g.][]{Bibring2006}{Murchie2009}. An understanding of the
depositional environment and alteration history of the layered sulfate
unit can help trace the evolution of the early Martian surface
environment. We assess detailed structural, mineralogical, and
morphological observations of the layered sulfates against a range of
possible formation mechanisms (e.g.~lava flow, ash fall, lacustrine
deposition). The layered sulfates were likely deposited as bedded
sediments gently draping the Noachian basement, and underwent diagenetic
volume-loss fracturing and acid-sulfate alteration.

\begin{Figure}
 \centering
 \includegraphics[width=\linewidth]{test-fixtures/lpsc-abstract/overview-map.pdf}
 \captionof{figure}{\textit{Regional overview of layered sulfates assembled from morphology,
mineralogy, and thermal inertia. The study area is centered on
\emph{75ºE, 15ºN}.
}}
 \label{fig:overview-map}
 \vspace{-5pt}
\end{Figure}

\section{A localized deposit}\label{a-localized-deposit}

The extent of the layered sulfates is mapped using HiRISE, CTX, and HRSC
imagery and derived elevation models {[}Figure
\ref{fig:overview-map}{]}. Morphologic mapping is refined with CRISM
mineral identifications and thermal inertia data. Based on the elevation
of bounding surfaces, the layered sulfates are up to
\textasciitilde{}400 m thick, tapering laterally to the northwest and
southeast towards the edges of the study area.
\end{multicols}

\begin{Figure}
\centering
\setlength\fboxsep{0pt}
\includegraphics[width=\textwidth]{test-fixtures/lpsc-abstract/nf6-bedding.pdf}
\captionof{figure}{\textit{Bedding orientations (95\% CI in dip--dip azimuth space, using method
from \autocite{Quinn2017MethodLPSC}) measured on HiRISE stereo pair
\emph{ESP\_021612\_1975}/\emph{ESP\_021757\_1975}. (\textbf{a})
Superposition of the Syrtis Major lavas atop a slope of layered sulfates
and a dipping basement pediment. (\textbf{b}) Sulfates in the north part
of the image dip southward beneath a flat lava flow. (\textbf{c}) In the
southern part of the image, layered sulfates dip \textasciitilde{}8º
southwest beneath south-dipping lava flow, creating an angular
unconformity.
}}
\label{fig:nf6-bedding}
\vspace{-15pt}
\end{Figure}

\begin{multicols}{2}

\small
\noindent{}This, along with the lack of similar units elsewhere in
Isidis Basin, suggests localized deposition or preferential shielding
from erosion within the study area.

\section{Draping deposition}\label{draping-deposition}

A new method for error analysis of bedding-orientation measurements
\autocite{Quinn2017MethodLPSC} is used to interpret the depositional
form of the layered sulfates. Dips are \textless{}10º over the entire
study area, but dip directions are variable, with low angle changes in
bedding evident at several-km scale. Dips greater than
\textasciitilde{}5º often loosely correspond to dipping basement
paleotopography, suggesting deposition on a tilted surface.

Figure \ref{fig:nf6-bedding} shows a stratigraphy with layered sulfates
beneath the capping Syrtis major lavas. Bedding orientations in the
sulfate are consistent at \textasciitilde{}4 km scale, but change at the
northern part of the image. In both areas, bedding is distinct from the
dip of the Syrtis Major lavas, showing a significant angular
unconformity that marks an erosional hiatus prior to lava emplacement.

\section{Volume-loss fracturing and
alteration}\label{volume-loss-fracturing-and-alteration}

Polygonal ridges at \textasciitilde{}500 m scale (Figure
\ref{fig:overview-map}) are key markers of alteration history of the
layered sulfates. These ridges dominantly intersect at right angles but
have no preferred orientation, and are often subvertical. In some areas,
localized bedding-orientation variability (up to \textasciitilde{}5º)
coincides with boxwork domains. These features are characteristic of
volume-loss fracturing without a regional stress field, analogous to 3D
polygonal faulting on Earth \autocite{Goulty2008}. Dewatering of
saturated sediments during diagenesis is a likely model for fracture
formation.

CRISM detections of jarosite along fracture traces, along with
isopachous fills visible from HiRISE imagery, show that these volume
loss-fractures were mineralized by acid-sulfate fluids. Variable
penetration of fluids leads to a wide range of fracture expression and
erosional resistance. Fracture fills grade into a bright-toned
alteration halo beneath the capping Syrtis Major lavas (Figure
\ref{fig:nf6-bedding}a), underscoring the association of fluid
mineralization with the overriding lava flows.

\section{Conclusion}\label{conclusion}

The layered sulfates at northeast Syrtis major represent a significant
sedimentary package that was both deposited and significantly eroded
during the Noachian--Hesperian transition, prior to capping by Syrtis
major lavas. The sedimentary origin suggested by the rhythmic bedding
and recessive erosional character is confirmed by the

\begin{Figure}
 \centering
 \includegraphics[width=\linewidth]{test-fixtures/lpsc-abstract/sulfate-emplacement-history.pdf}
 \captionof{figure}{\textit{A model emplacement history for the layered sulfates, emphasizing the
separability of sedimentation, volume-loss fracturing, and fluid
alteration. The relative order of fracturing and lava flows is
uncertain; volume-loss could occur in response to, or after, lava
emplacement. In this alterative scenario, the panels \emph{(2,3,4)}
would be reordered \emph{(3,2,4)}.
}}
 \label{fig:sulfate-emplacement-history}
 \vspace{-5pt}
\end{Figure}

\noindent{} flat to gently draping layering and pattern of volume-loss
fracturing. The regional layering style and pattern of volume-loss
fractures is characteristic of basinal deposition. After deposition, the
deposits underwent a multistage history of volume loss and fluid
mineralization. Overall, the layered sulfates show a significant
presence of water at the surface of Mars during the Noachian--Hesperian
transition.
